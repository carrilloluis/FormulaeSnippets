\documentclass[10pt, a5paper]{article}

\usepackage[left=1.5cm,right=1.5cm]{geometry}
\usepackage[utf8]{inputenc}
\usepackage[spanish]{babel}
\usepackage{amssymb, amsmath, amsbsy}
\usepackage{enumerate}
\usepackage{verbatim}
\usepackage{chemfig}
\usepackage{mathpazo}

\renewcommand{\familydefault}{\sfdefault}

\begin{document}
% Dilatación del tiempo
% Cuanto más rápido te mueves en el espacio,
% más lento te mueves a través del tiempo.

$$
T\text{'} = \dfrac{T}{\sqrt{1 - \dfrac{v^2}{c^2} }}
$$

% Donde:
% T = Intervalo de tiempo entre dos eventos locales para un observador en un marco inercial?
% T' = Intervalo de tiempo entre esos mismos eventos, medido por otro observador, moviéndose inercialmente con velocidad 'v' respecto al observador anterior.

$$
\Delta t\text{'} = \dfrac{\Delta t}{\sqrt{1 - \dfrac{v^2}{c^2} }}
$$

% https://es.wikipedia.org/wiki/Dilataci%C3%B3n_del_tiempo
\end{document}