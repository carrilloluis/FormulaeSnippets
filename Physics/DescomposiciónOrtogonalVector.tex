\documentclass[12pt, a5paper]{article}

\usepackage{geometry}
\usepackage[utf8]{inputenc}
\usepackage[spanish]{babel}
\usepackage{amsmath, amssymb}
%\usepackage{stix} % , bm}
% \usepackage{fourier} % \usepackage{enumerate}
% \usepackage[garamond]{mathdesign}
%\usepackage{garamondx}
\usepackage{tikz}
% \usetikzlibrary{arrows}
\usetikzlibrary{babel}
% \usepackage{verbatim}

%\usepackage[T1]{fontenc}
%\geometry{textwidth=7cm}
%\usepackage{tgbonum}
% \usepackage{times} % Si funciona!!!
%\usepackage{Lucida Sans Typewriter}
%\usepackage{garamond}
\usepackage{mathpazo} % GOOOD!!!!

\usepackage[pdftex]{hyperref}
\hypersetup{ pdftitle = {Vectores},  pdfauthor = {Luis Carrillo Gutiérrez}, pdfsubject= {vector, componentes, descomposición ortogonal de un vector} }

\renewcommand{\familydefault}{\sfdefault}

\begin{document}

\subsection*{Descomposición Ortogonal de un vector}
\begin{tikzpicture}
\draw [thick, ->] (0, -1) -- (0, 3);
\draw [thick, ->] (-1, 0) -- (6, 0);
\draw (0, 3) node[left]{{\tt Y}};
\draw (6, 0) node[below]{{\tt X}};
\draw [very thick, ->] (0, 0) -- (5, 2.5);
\draw [gray, dashed] (0, 2.5) -- (5, 2.5);
\draw [gray, dashed] (5, 0) -- (5, 2.5);

\draw (2, 1.4) node{$\vec{R}$};
\draw (3, 0) node[below]{$\vec{R_x}$};
\draw (0, 1.4) node[left]{$\vec{R_y}$};

\draw (0.6, 0.0) arc [radius=0.6, start angle=0, end angle=25];
\draw (0.95, 0.25) node{$\theta$};
\end{tikzpicture}

$$
\vec{R} = \vec{R_x} + \vec{R_y} \qquad : \qquad \vec{R_x} \perp \vec{R_y} % \vec{R_x}
$$

$$
\left.\begin{array}{c} R_x = R \cdot \cos \theta \\ R_y = R \cdot \sen \theta\end{array}\right\rbrace R^2 = {R_x}^2 + {R_y}^2
$$


\end{document}