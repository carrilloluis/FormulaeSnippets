\documentclass[12pt, a5paper]{article}

\usepackage{geometry}
\usepackage[utf8]{inputenc}
\usepackage[spanish]{babel}
\usepackage{amsmath, amssymb}
%\usepackage{stix} % , bm}
% \usepackage{fourier} % \usepackage{enumerate}
% \usepackage[garamond]{mathdesign}
%\usepackage{garamondx}
\usepackage{tikz}
% \usetikzlibrary{arrows}
\usetikzlibrary{babel}
% \usepackage{verbatim}

%\usepackage[T1]{fontenc}
%\geometry{textwidth=7cm}
%\usepackage{tgbonum}
% \usepackage{times} % Si funciona!!!
%\usepackage{Lucida Sans Typewriter}
%\usepackage{garamond}
\usepackage{mathpazo} % GOOOD!!!!

\usepackage[pdftex]{hyperref}
\hypersetup{ pdftitle = {Vectores},  pdfauthor = {Luis Carrillo Gutiérrez}, pdfsubject= {vector, componentes, descomposición ortogonal de un vector} }

\renewcommand{\familydefault}{\sfdefault}

\begin{document}


\section*{Vector}
\subsection*{Suma vectorial}
\empty \indent \\
\begin{tikzpicture}
\draw (4.6, 2) node{$\Rightarrow$};
\draw [very thick, ->] (0, 0) -- (1.6, 3);
\draw (0.75, 1.8) node[left]{$\vec{A}$};
\draw [very thick, ->] (0, 0) -- (4, 0);
\draw (2, -0.35) node{$\vec{B}$};
\draw (0.55, 0.4) node{$\theta$};
% ->[dashed]
\draw [very thick, ->] (5, 0) -- (6.6, 3);
\draw [very thick, ->] (5, 0) -- (9, 0);
\draw [dashed] (9, 0) -- (10.6, 3);
\draw [dashed] (6.6, 3) -- (10.6, 3);
\draw [dashed] (9, 0) -- (11, 0);
\draw [dashed] (10.6, 0) -- (10.6, 3);
\draw [gray] (0.4, 0.0) arc [radius=0.4, start angle=0, end angle=60];
% jerarcas??
\draw (5.75, 1.8) node[left]{$\vec{A}$};
\draw (7, -0.35) node{$\vec{B}$};
\draw [thick, ->] (5, 0) -- (10.6, 3);
\draw (5.6, 0.0) arc [radius=0.6, start angle=0, end angle=30];
\draw (5.9, 0.2) node{$\alpha$};
\draw [gray] (9.4, 0.0) arc [radius=0.4, start angle=0, end angle=60];
\draw (9.55, 0.4) node{$\theta$};
\draw (7.6, 1.8) node{$\vec{R}$};
\end{tikzpicture}

$$
\vec{R} = \vec{A} + \vec{B}
$$

$$
% \norm{
|\vec{R}|  = \sqrt{ |\vec{A}|^2 + |\vec{B}|^2 + 2 \cdot |\vec{A}| \cdot |\vec{B}| \cdot \cos \theta}
$$

$$
R = \sqrt{ A^2 + B^2 + 2 \cdot A \cdot B \cdot \cos \theta}
$$

$$
\alpha = \arctan{\dfrac{A \cdot \sen \theta}{B + A \cdot \cos \theta}}
$$


\end{document}