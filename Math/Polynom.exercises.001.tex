\documentclass[12pt, a5paper]{article}

\usepackage[top=1cm]{geometry}
\usepackage[utf8]{inputenc}
\usepackage[spanish]{babel}
\usepackage{amssymb, amsmath, amsbsy}
\usepackage{mathpazo}

\renewcommand{\familydefault}{\sfdefault}

\begin{document}
\subsection*{Trabajando en casa}

\begin{itemize}
\item{% 1
Si $5x^{a + 3}y^{8}\quad\wedge\quad \dfrac{3}{5}x^{8}y^{b + 2}$, son términos semejantes, calcula ``$a + b$''.
}
\item{
Calcula el grado relativo y el absoluto en cada caso:
$$
P(x, y, z) = 3x^{2}y^{5}z^{3} + \pi x^{4}y^{3}z^{6} + 3x^7 
$$
$$
R(x; y) = 2x^{3}y^{3} + 5x^{10} + 2^{3}y^{6}
$$
}
\item{
Sea : $P(x - 1) = 2x - 3$, calcula $P(3) - P(-2)$.
}
\item{
Si $P(x - 3) = 5x + 2$, calcula $P(2x - 1)$.
}
\item{% 5
Si $P(x + 3) = 2x - 5$, calcula $P(x - 5)$.
}
\item{
Suma : $8x^{a + b}y^{16} \quad\wedge\quad bx^{8}y^{a - b}$
}
\item{
Si el grado absoluto de $R$ es 11, determine el valor de ``$n$''.
$$
R(x; y) = x^{3n - 1}y^{n} - 2x^{2n - 2}y^{2n} + x^{n + 3}y^{3n}
$$
}
\item{
Si : $P(x) = (x - 1)^{47} + (x + 2)^{3} + x - 3 + a$, y a su término independiente es -15, calcula la suma de coeficientes de $P(x)$.
}
\item{% 9
Si: $P(x) = (x-1)^{42} + (x + 1)^4 + x + 2 + m$, y su término independiente es 10, calcula la suma de coeficientes de $P(x)$.
}
\item{% 10
Sea $F(x)$ un polinomio que cumple con $F(x+1) = 3F(x) - 2F(x-1)$, además:
$F(4) = 1 \wedge F(6)=4$. Calcula $F(5)$.
}
\item{% 11
En el siguiente polinomio: $P(x,y) = x^{a}y^{b - 1} + x^{a + 1}y^{b} - x^{a - 2} + x^{a + 3}y^{b + 1}$. Donde $G.R.(x) = 10$, $G.A. = (P) = 113$. Determina el $G.R.(y)$.
}
\item{% 12 
Calcula el valor de $n$ en el siguiente polinomio: $P(x) = 2x^{n-7} + \sqrt{3} x^{\frac{n}{3}} - 5x^{10 - n}$.
}
\item{% 13
Calcula el valor de n en el siguiente polinomio. $P(x) = 7x^{n-22} + \sqrt{2}x^{\frac{n}{5}} + 13x^{29 -n}$
}
\item{% 14
Si $P(x) = ax^2 + bx + c$ Además: $P(0) = 3$, $P(-1) = 7$; $P(1) = 1$, Calcula $P(2)$.
}
\end{itemize}
\end{document}







