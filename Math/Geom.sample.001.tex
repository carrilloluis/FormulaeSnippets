\documentclass[12pt, a5paper]{article}

\usepackage{geometry}
\usepackage[utf8]{inputenc}
\usepackage{amssymb, amsmath, amsbsy}
\usepackage{mathpazo}

\renewcommand{\familydefault}{\sfdefault}

\begin{document}
\subsection*{Geometría 001}
\noindent En la siguiente figura el lado $\overline{AC}$ es bisectriz del ángulo $\angle BAD$. Determina los ángulos interiores de los $\triangle ABC$ y $\triangle ACD$ sabiendo que $\angle BAC = y + 8^{\circ}$, $\angle CAD = x + 13^{\circ}$, $\angle ABC = 3x - 6^{\circ}$  y $\angle ACD = \dfrac{10}{3} y + 7^{\circ}$.

\end{document}
