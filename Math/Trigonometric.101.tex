\documentclass[12pt, a5paper]{article}

\usepackage{geometry}
\usepackage[utf8]{inputenc}
\usepackage[spanish]{babel}
% \usepackage{amsmath} , amsbsy
\usepackage{amssymb, amsmath}
\usepackage{tikz}

\renewcommand{\familydefault}{\sfdefault}
\DeclareMathSizes{14}{14}{14}{14}

\begin{document}
	\paragraph{Función} -- Es el conjunto de pares ordenados de números (x, y) tales que dos parejas distintas no tienen el mismo primer elemento.
	
	\paragraph{Dominio} -- Conjunto de todos los posibles valores de ``x".
	
	\paragraph{Rango} -- Conjunto de todos los posibles valores de ``y".
	
	\subsection*{Gráfica de una función}
	En un plano cartesiano, es el conjunto de todos los puntos (x, y) que pertenecen a la función.
	
	\paragraph{Notación} -- Se denota : $y = f(x)$
	y se lee ``f" de ``x" % leáse
	\subsection*{Funciones especiales}
	\begin{itemize}
		\item{Función Identidad
			$$y = x$$
		}
		\item{Función líneal (o Línea recta)
			$$y = mx + b$$
		}
		\item{Función cuadrática (ó Parábola)
			$$y = ax^{2} + bx + c$$
		}
		\item{Ecuación de la circunferencia con centro en el origen de coordenadas
			$$x^{2} + y^{2} = r^{2}$$
		}
	\end{itemize}
	
	\subsection*{Funciones trascendentes}
	Son aquellas no algebraicas tales como las trigonométricas, logarítmicas o exponenciales.
	
	\subsection*{Variación de las funciones trigonométricas}

	\paragraph{Sinusoide} : $y = \sen x$
	\begin{center}
		\begin{tikzpicture}[xscale=0.01]
			\draw [domain=0:360,samples=100] plot(\x, { sin(\x ) } );
		\end{tikzpicture}
	\end{center}  
	
	\paragraph{Cosinusoide} : $y = \cos x$
	\begin{center}
		%\begin{tikzpicture}[xscale=0.01]
			% \draw [domain=0:360,samples=100] plot(\x, { sin(0.9 * \x ) } );
		%\end{tikzpicture}
	\end{center}
	
	
	\section{Razones trigonométricas de Ángulos Cuadrantales}
	
	\section*{Identidades trigonométricas}
	
	\subsubsection*{Pitagóricas}
	\begin{eqnarray*}
		\sen^{2} a + \cos^{2} a & = & 1 \\
	\end{eqnarray*}
	
	\subsubsection*{Razón o cociente}
	\begin{eqnarray*}
		\tan a = \dfrac{\sen a}{\cos a} \\
		\cotg a = \dfrac{\cos a}{\sen a}
	\end{eqnarray*}
	
	
	\subsubsection*{Recíprocas}
	\begin{eqnarray*}
		\sen a \cdot \csc a & = & 1 \\
		\cos a \cdot \sec a & = & 1 \\
		\tan a \cdot \cotg a & = & 1 
	\end{eqnarray*}
	
	\subsection*{Funciones Trigonométricas de la suma y diferencia de 2 arcos/ángulos}
	
	\begin{eqnarray*}
		\sen(a \pm b) & = & \sen a \cdot \cos b \pm \sen b \cdot \cos a \\
		\cos(a \pm b) & = & \cos a \cdot \cos b \mp \sen a \cdot \sen b \\
		\tan (a \pm b) & = & \dfrac{\tan a \pm \tan b}{1 \mp \tan a \cdot \tan b} \\
		\cotg (a \pm b) & = & \dfrac{\cotg a \cdot \cotg b \mp 1}{\cotg b \pm \cotg a}
	\end{eqnarray*}
	
	\subsection*{Funciones Trigonométricas de un ángulo doble}
	\begin{eqnarray*}
		\sen 2\theta & = & 2 \cdot \sen \theta \cdot \cos \theta \\
		\cos 2\theta & = & 2 \cdot cos^{2} \theta - 1 \\
		\empty & = & 1 - 2 \cdot \sen^{2} \theta \\
		\tan 2\theta & = & \dfrac{2 \cdot \tan \theta}{1 - \tan^{2} \theta}
	\end{eqnarray*}
	
	\subsection*{Funciones Adicionales de un ángulo doble}
	
	
	
	\subsection*{Funciones Trigonométricas de la mitad de un ángulo/arco}
	El signo depende del cuadrante donde se encuentra $\dfrac{\theta}{2}$.
	
	\begin{eqnarray*}
		\sen \dfrac{\theta}{2} & = & \pm \sqrt{\dfrac{1 - \cos \theta}{2}} \\
		\cos \dfrac{\theta}{2} & = & \pm \sqrt{\dfrac{1 + \cos \theta}{2}} \\ 
		\tan \dfrac{\theta}{2} & = & \pm \sqrt{\dfrac{1 - \cos \theta}{1 + \cos \theta}} \\ 
		\tan \dfrac{\theta}{2} & = & \dfrac{-1 \pm \sec \theta}{\tan \theta} \\ 
		\cotg \dfrac{\theta}{2} & = & \pm \sqrt{\dfrac{1 + \cos \theta}{1 - \cos \theta}} \\ 
	\end{eqnarray*}
	
	
	\section*{Relaciones entre los elementos de un triángulo}
	\begin{tikzpicture}
		
	\end{tikzpicture}
	
	\paragraph{Ley de senos}
	$$
	2R = \dfrac{a}{\sen A} = \dfrac{b}{\sen B} = \dfrac{c}{\sen C}
	$$
	
	\paragraph{Ley de cosenos}
	$$
	a^{2} = b^{2} + c^{2} + 2bc \cdot \cos A
	$$
	
	\paragraph{Ley de tangentes}
	$$
	\dfrac{a + b}{a - b} = 
	$$
	Radio de la circunferencia inscrita
	Superficie
	
	
\end{document}

\begin{comment}
	\documentclass{minimal}
\usepackage{tikz,pgfplots}

\begin{document}
\begin{tikzpicture}[domain=0:4]
    \draw[very thin,color=gray] (-0.1,-1.1) grid (3.9,3.9);
    \draw[->] (-0.2,0) -- (4.2,0) node[right] {$x$};
    \draw[->] (0,-1.2) -- (0,4.2) node[above] {$f(x)$};
    \draw[color=red]    plot (\x,\x)    node[right] {$f(x) =x$};
    \draw[color=blue]   plot (\x,{sin(\x r)})   node[right] {$f(x) = \sin x$};
    \draw[color=orange] plot (\x,{0.05*exp(\x)}) node[right]
        {$f(x) = \frac{1}{20} \mathrm e^x$};
\end{tikzpicture}
\end{document}



\documentclass{article}
\usepackage{pgfplots}

\begin{document}
\begin{tikzpicture}
    \begin{axis}[domain=0:1,legend pos=outer north east]
    \addplot {sin(deg(x))}; 
    \addplot {cos(deg(x))}; 
    \addplot {x^2};
    \legend{$\sin(x)$,$\cos(x)$,$x^2$}
    \end{axis}
\end{tikzpicture}
\end{document}



How can we plot the following three functions

    f(x) = sin(x)
    k(x) = cos(x)
    u(x) = x²

for x ∈ [0,1]


    \documentclass{minimal}
    \usepackage{tikz}
     
    \begin{document}
       \begin{tikzpicture}
         \draw (-0.5,0) -- (11,0) (0,-1.5) -- (0,1.5);
         \draw plot[domain=0:9.4248,smooth] (\x,{sin(\x r)});
         \draw plot[domain=0:10.4719,smooth] (\x,{sin(0.9*\x r)});
         \fill[red] (4.1888,{sin(4.1888 r)}) circle (2pt);
       \end{tikzpicture}
    \end{document}
    
    
\end{comment}