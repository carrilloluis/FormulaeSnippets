\documentclass[10pt, a5paper]{article}

\usepackage[left=1.5cm,right=1.5cm]{geometry}
\usepackage[utf8]{inputenc}
\usepackage[spanish]{babel}
\usepackage{amssymb, amsmath, amsbsy}
\usepackage{enumerate}
\usepackage{verbatim}
\usepackage{chemfig}
\usepackage{mathpazo}

\renewcommand{\familydefault}{\sfdefault}

\begin{document}
% velocidad mínima para escapar a la fuerza de gravedad = 40300 Km/h
% Mientras más grande la masa y menor el radio de un cuerpo celeste esférico,
% mayor será la velocidad necesaria para escapar de su influencia gravitacional.

$$
v_{e} = \sqrt{\dfrac{2 G M}{r}}
$$
% V_{e} Velocidad de Escapa
% M masa del cuerpo (sistema masivo)?
% G Constante de gravitación universal? (6.672×10^{−11} N m²/kg²).
% r radio | distancia entre los centros de masa?

$$
v_{e} = \sqrt{2gr}
$$

% g es la intensidad del campo gravitatorio. En la superficie de la Tierra se toma g = 9.81 m/s2.

% la velocidad de escape media desde el nivel del mar es de 11,19 km/s (kilómetros por segundo), lo que equivale a 40280 km/h (kilómetros por hora).

\end{document}