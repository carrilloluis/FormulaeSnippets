\documentclass[12pt, a5paper]{article}

\usepackage[spanish]{babel}
\usepackage{geometry}
\usepackage{amssymb, amsmath, amsbsy}
\usepackage{hyperref}
\usepackage{mathpazo}

\renewcommand{\familydefault}{\sfdefault}

\begin{document}
\subsection*{Identidades trigonómetricas}
% tangente de ángulo = racional seno / coseno
$$
\tan x = \dfrac{\sen x}{\cos x}
$$

$$
\arcsen x = \arctg \dfrac{x}{\sqrt{1 - x^2}}
$$

$$
\arccos x = \arctg \dfrac{\sqrt{1 - x^2}}{x}
$$

$$
\pi = 4 \arctg 1
$$

$$
\log_{10} x = \dfrac{\ln x}{\ln 10}
$$

$$
x^y = \exp (y \cdot \ln x)
$$

$$
\sinh x = \dfrac{ e^x - e^{-x}}{2}
$$

$$
\cosh x = \dfrac{ e^x + e^{-x}}{2}
$$

$$
\tanh x = \dfrac{\sinh x}{\cosh x}
$$

$$
\arcsen\negthinspace\text{h}\;x = \ln (x + \sqrt{x^2 + 1})
$$

$$
\arccos\negthinspace\text{h}\;x = \ln (x + \sqrt{x^2 - 1})
$$

$$
\arctan\negthinspace\text{h}\;x = \dfrac1{2} \ln \dfrac{1 + x}{1 - x}
$$
\end{document}